\chapter{Group Actions}
\underline{Assume}: $s \in S$ is a set; $G = (G, \star)$ is a group. \\
Identity element under $\star$ in $G$: ${g}_{id, \star} = 1$
\section{Group Action}
\begin{defn}[Group Action of $G$ on $S$]
	A map $G \times S \to S$ satisfying:
	\[\forall {g}_{1}, {g}_{2} \in G \; \; \; \forall s \in S\]
	\begin{enumerate}
		\item $ {g}_{1}  ( {g}_{2}  s ) = ( {g}_{1} \star {g}_{2} ) s $
		\item $ 1s = s$ 
	\end{enumerate}
Every element of the group acts as a permutation on the set. Many different actions of a group G on a given set S can be constructed.
\end{defn}

\section{Centralizers, Centers, and Normalizers}
\begin{defn}[Centralizer of $S$ in $G$]
	Set of elements of G which commute with every element of S.
	\[ {C}_{G}(S) = \{g \in G \mid g  s  {g}^{-1} = s;  \;\; \forall s \in S \}\]
\end{defn}
\begin{defn}[Normalizer of $S$ in $G$]
	Set of elements and their inverses from G such that when applied together to a set S they return the set S.
	\[{N}_{G}(S) = \{g \in G \mid gS{g}^{-1} = S\}\]
	The centralizer is a subgroup of the normalizer:
	\[{C}_{G}(S) \leq {N}_{G}(S)\]
\end{defn}
\begin{defn}[Center of $G$]
	Set of elements commuting with every element of $G$.
	\[Z(G) = \{g \in G \mid gx = xg;  \;\; \forall x \in G\}\]
\end{defn}


\section{Stabilizers and Kernels}
\begin{defn}[Stabilizer of $s$ in $G$]
	Set of group elements that act on an element of the set and return the same set element:
	\[{G}_{s} = \{g \in G \mid g s = s\}\]
	i.e. the set of group elements that act trivially on some element of the set.
\end{defn}
\begin{defn}[Kernel of the Action]
	Set of group elements that act on every element of the set and return that same set element. (could be identity map).
	\[ {A}_{ker\varphi} = \{g \in G \mid g s = s;  \;\; \forall s \in S\}\]
	i.e. The set of group elements that act trivially on all elements of the set.
\end{defn}
\begin{defn}[Faithful action]
	If an action's kernel is the identity. Any action with more elements than just the identity is \underline{not} faithful.
\end{defn}

\section{Orbits}
\underline{Assume}: $\mathcal O$ is the symbol for a set of orbits, meaning a set with one or more elements that are what the set element can get mapped to. \\ 

So $x \in \mathcal O$ just means that some element under the group permutation gets mapped to that $x$.
\begin{defn}[Orbit of $s$ in $G$]
	The set of elements that a single element $s \in S$ maps to under the various permutations $g \in G$ of the group under the action.
	\[\{ga \mid g \in G\}\]
	i.e. All the places $s \in S$ can "go" under all the allowed symmetries of the group $G$.
\end{defn}
\begin{defn}[Transitive Orbit]
	If $s \in S$ only maps to one element under $g \in G$.
\end{defn}

\subsection{Conjugacy Classes of $G$}
\begin{defn}[Conjugacy Classes of G]
	The orbits $\mathcal O$ of G acting on itself by conjugation.
\end{defn}

\subsection{Conjugation}
\underline{Assume}: $G$ is acting on itself.
\begin{defn}[Conjugate elements in $G$]
	If two group elements $a, b \in G$ have an associated element $g \in G$ such that:
	\[b = ga{g}^{-1} \]
	i.e. iff they are in the same orbit of G acting on itself by conjugation.
\end{defn}
\begin{defn}[Conjugate Sets in $G$]
	two subsets, $T \subseteq G$ and $S \subseteq G$ are conjugate if there is some $g \in G$ such that:
	\[ T = gS{g}^{-1} \] 
	i.e. iff they are in the same orbit of G acting on its subsets by conjugation.
\end{defn}

\section{Permutation Representations}
\underline{Assume}: $A \not = \emptyset$, $G$ is a group, and $\varphi$ a homomorphism, and $G$ acts on $A$.
\begin{defn}[Permutation Representation]
	Any $\varphi: G \to {S}_{A}$. \\
	i.e. Any homomorphism of a group $G$ into the symmetric group ${S}_{A}$.
\end{defn}
An action of $G$ on $A$ \textit{induces} the associated permutation representation of $G$. It's similar to the matrix representation of a linear transformation.