\chapter{Domains}
\section{Norm}
\underline{Assume} $R$ is an integral domain.
\begin{defn}[Norm]
	Any function $N: R \to {Z}^{+} \cup \{0\}$ with $N(0) = 0$.
\end{defn}
\begin{defn}[Positive Norm]
	If $N(a) > 0$ for $a \not = 0$
\end{defn}
\section{Euclidean Domains: E.D.'s}
\begin{defn}[Euclidean Domain]
	If there is a norm N on R such that for any two elements $a, b \in R$ with $b \not = 0$ then $\exists$ $q, r \in R$ with:
	\[a = bq + r\]
	with either:
	\[r = 0\]
	\[OR\]
	\[N(r) < N(b)\]
\end{defn}
Every Ideal in a Euclidean Domain is a principal, therefore:
\underline{ every Euclidean Domain is a P.I.D.}

\section{Principal Ideal Domains: P.I.D.'s}
\begin{defn}[P.I.D.]
	Integral domain R in which every ideal is principal.
\end{defn}
\underline{Every P.I.D. is NOT a Euclidean Domain.}

\section{Unique Factorization Domains: U.F.D.'s}
\underline{Every P.I.D. is a Unique Factorization Domain.}
