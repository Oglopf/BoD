\chapter{Rings}
\underline{Assume}: $R$ and $S$ are Rings and $\varphi$ is a map $\varphi: R \to S$. 


\section{Rings}
\textbf{Additive Identity}: ${r}_{id, +} = 0$ \\ 
\textbf{Multiplicative Identity}: ${r}_{id, \cdot } = 1$
\begin{defn}[Ring $R$]
	A set $R$ with two binary operations + and $\cdot$ satisfying:
	\begin{enumerate}
		\item (R, +) is Abelian group.
		\item $\cdot$ is associative in R:
		\[(a \cdot b) \cdot c = a \cdot (b \cdot c)\]
		\item Distributive laws hold in R:
		\[(a + b) \cdot c = (a \cdot c) + (b \cdot c)\]
		\[a \cdot (b + c) = (a \cdot b) + (a \cdot c) \]
	\end{enumerate}
\end{defn}

\begin{defn}[Commutative Ring]
	Ring R that is commutative under $\cdot$ multiplication.
\end{defn}

\begin{defn}[Ring with Identity ${R}_{1}$]
	If $\exists!$ $1  \in R$ satisfying:
	\[ 1  \cdot a = a \cdot 1 = a\]
\end{defn}
\underline{Note}: Typically $R$ is said to have $1 \not = 0$ in texts, meaning ${R}_{1}$ in my notes. 

\section{Division Rings, Integral Domains, Fields}
\begin{defn}[Division Ring] ${R}_{1}$ with elements:
	\[ {r}^{-1} \in R \; \; \; r \not = 0 \in R \] 
	such that:
	\[ r \cdot {r}^{-1} = {r}^{-1} \cdot r = 1\]
	i.e. every element has an inverse element under product relationship.
\end{defn}
Why would a black hole exist in a field then? That seems like a contradiction due to the singularity it implies.
\begin{defn}[Zero Divisor]
		An element $r \in R$ composed with $x \not = 0 \in R$ such that:
	\[ r \cdot x = x \cdot r = 0 \]
\end{defn}

\begin{defn}[Integral Domain]
	Any ${R}_{1}$ with \underline{no} zero divisors.
\end{defn}
\begin{defn}[Field]
	A set F together with two commutative binary operations $\cdot$ and + over F such that:
	\begin{enumerate}
		\item (F, +) is an Abelian group.
		\item $(F - \{ 0 \}, \cdot)$ is an Abelian group.
		\item The distributive laws hold $\forall a, b, c \in F$:
		\[a \cdot (b + c) = a \cdot b + a \cdot c\]
		\[(a + b) \cdot c = (a \cdot c) + (b \cdot c)\]
	\end{enumerate}
\end{defn}
\begin{prop}[Field by Integral Domain]
	Any finite integral domain is a field.
\end{prop}
\section{Subrings}
\begin{defn}[Subring Subgroup Language]
	A subgroup of $R$ that is closed under multiplication.
\end{defn}
\begin{defn}[Subring Subset Language]
	If $A \subseteq R$ satisfying:
	\begin{enumerate}
		\item $ A \not = \emptyset$
		\item A is closed under subtraction and multiplication.
	\end{enumerate} 
\end{defn}

\section{Ring Homomorphism, Isomorphism, and Kernel}
\begin{defn}[Ring Homomorphism]
	Any $R$, $S$ and $\varphi$ satisfying:
	\begin{enumerate}
		\item $\varphi(a + b) = \varphi(a)+\varphi(b)$ $\forall a,b \in R$.
		\item $\varphi(ab) = \varphi(a)\varphi(b)$ $\forall a,b, \in R$.
	\end{enumerate}
\end{defn}

\begin{defn}[Ring Isomorphism]
	A bijective ring homomorphism.
\end{defn}

\begin{defn}[Kernel of Ring Homomorphism $ker\varphi$]
	The set of elements of $R$ that map to ${e}_{id} \in S$.
	\[ker\varphi = \{r \mid \varphi: r \in R \to {e}_{id}\in S \}\]
\end{defn}

\section{Ideals: Principal, Maximum, Prime}
\underline{Assume}: \\ 
$P, M, I$ represent ideals. \\
$S$ is an arbitrary ring. \\
$R$ is a commutative ring. \\
${R}_{1}$ is a commutative ring with $1 \not = 0$. \\

Analogous to subgroups for groups.
\begin{defn}[Ideal $I$]
	Let $I \subseteq S$ and $s \in S$ satisfying:
	\begin{enumerate}
		\item $sI = \{sa \mid a \in I\}$ and $ Is = \{as \mid a \in I\}$
		\item Left or Right ideal if:
		\begin{enumerate}
			\item $I$ is a subring of $R$.
			\item $I$ is closed under left/right multiplication i.e. $sI \subseteq I$;  $\;\;\forall s \in S$.
		\end{enumerate}
	\end{enumerate}
Then $I$ is a \textbf{left/right ideal}.
If both then \textbf{ideal}.
\end{defn}
\begin{defn}[Principal Ideal]
	An ideal generated by a \underline{single} element.
\end{defn}
\begin{defn}[Maximum Ideal]
	If $M \not = S$ and the only ideals containing M are M and S.
\end{defn}
So, $M$ is not contained in any other proper ideal of $S$.
\begin{defn}[Prime Ideal]
	If $P \not = R$ and $\forall a, b \in R$ the product $ab \in P$ implies \underline{at least one} of a and b is an element of P.
	
\end{defn}